% TEMPLATE for Usenix papers, specifically to meet requirements of
%  USENIX '05
% originally a template for producing IEEE-format articles using LaTeX.
%   written by Matthew Ward, CS Department, Worcester Polytechnic Institute.
% adapted by David Beazley for his excellent SWIG paper in Proceedings,
%   Tcl 96
% turned into a smartass generic template by De Clarke, with thanks to
%   both the above pioneers
% use at your own risk.  Complaints to /dev/null.
% make it two column with no page numbering, default is 10 point

% Munged by Fred Douglis <douglis@research.att.com> 10/97 to separate
% the .sty file from the LaTeX source template, so that people can
% more easily include the .sty file into an existing document.  Also
% changed to more closely follow the style guidelines as represented
% by the Word sample file. 
% This version uses the latex2e styles, not the very ancient 2.09 stuff.
\documentclass[10pt,twocolumn]{report}
\usepackage{graphicx,epsfig,endnotes}
\begin{document}

%don't want date printed
% \date{}

%make title bold and 14 pt font (Latex default is non-bold, 16 pt)
\title{\Large \bf DataSeries User Guide}

% \numberofauthors{0}
%for single author (just remove % characters)
% \author{
% \alignauthor Eric Anderson\\
% \affaddr{HP Labs}\\
% \affaddr{Palo Alto, CA}\\
% \email{eric.anderson4@hp.com}
% \alignauthor Martin Arlitt\\
% \affaddr{HP Labs}\\
% \affaddr{Palo Alto, CA}\\
% \email{martin.arlitt@hp.com}
% \alignauthor Charles B. Morrey III\\
% \affaddr{HP Labs}\\
% \affaddr{Palo Alto, CA}\\
% \email{brad.morrey@hp.com}
% \and
% \alignauthor Alistair Veitch\\
% \affaddr{HP Labs}\\
% \affaddr{Palo Alto, CA}\\
% \email{alistair.veitch@hp.com}
%\and
%{\rm Second Name}\\
%Second Institution
% copy the following lines to add more authors
% \and
% {\rm Name}\\
%Name Institution
% } % end author

\maketitle

% Use the following at camera-ready time to suppress page numbers.
% Comment it out when you first submit the paper for review.
%\thispagestyle{empty}

\section{Introduction}

...

The datasets we have collected were collected at a feature animation
(movie) company\footnote{The customer name will remain blinded as part
of the agreement to publish the traces} during two other projects.
The first dataset was collected between fall 2003 and spring 2004.  We
returned later to collect data in fall 2006-spring 2007 using an
improved version of our technology.  We collected data using mirror
ports on the customer's, and we mirrored a variety of different places
in the network.  The customers network design is straightforward: They
have a redundant core set of routers, an optional mid-tier of switches
to increase the effective port count of the core, and then a
collection of edge switches that each cover one or two racks of
rendering machines.  Most of our traces were taken by mirroring at
those rendering clients.

...


\chapter{Generic Programs}\label{chap:programs}

There are a number of existing programs present in DataSeries that can
be useful in working with DataSeries files.  We describe when one of
these programs would be used, and sketch the general usage of each
program here.  Detailed usage information on a program can be found on
that programs manual page.

Usually when you download DataSeries files from a remote repository,
the files will have been compressed with bz2\cite{BZIP2} compression.
This is done so that the files will be a small as possible during
transfer. However, bz2 is slow to decompress, and so it is useful to
convert the downloaded files to a faster decompression algorithm, and
sacrifice increased local storage space space.  The {\tt
dsrepack}\ref{program:dsrepack} program will perform this conversion.

The next most common operation is to take a quick look at the
downloaded DataSeries files.  The {\tt ds2txt}\ref{program:ds2txt}
program will take a DataSeries file and convert it to text so that the
type descriptions for the data stored in the file can be seen, and so
that a few of the sample lines can be examined.

A better understading of the data in the file may be gained by looking
at some statistics in the downloaded files.  This can be done by using
the {\tt dsstatgroupby}\ref{program:dsstatgroupby} program to
calculate statistics over various combinations of columns grouped by
another column.  This program can be looked at as a very restrictive
version of the SQL select statement.

Finally, before running a large number of analysis programs over the
downloaded files, it is usually useful to build an index over the
downloaded files.  This step can be performed by running {\tt
dsextentindex}\ref{program:dsextentindex}.  This generates an index
which can efficiently return the extents\ref{file-format:extent}
(sub-file chunks) that contain some amount of the desired data for the
analysis.

\section{{\tt dsrepack} -- re-compressing and merging DataSeries files}
\label{program:dsrepack}

\section{{\tt ds2txt} -- converting DataSeries files to text}
\label{program:ds2txt}

\section{{\tt dsstatgroupby} -- calculating statistics over input data}
\label{program:dsstatgroupby}

\section{{\tt dsextentindex} -- indexing DataSeries files}
\label{program:dsextentindex}



\chapter{Writing an analysis modules}\label{chap:analysis-module}

The most common operation in DataSeries is that you want to perform an analy

\chapter{Existing modules}\label{chap:existing-modules}

\chapter{Writing a conversion program}\label{chap:conversion-programs}

\chapter{Specific analysis programs}\label{chap:specific-analysis}


\chapter{File Format}\label{chap:file-format}

\section{Extents}\label{file-format:extent}

...


\bibliography{references}
\bibliographystyle{abbrv}
\end{document}

