\documentclass{acm_proc_article-sp}

\begin{document}
% replace the name to unblind
\newcommand{\DataSeries}{DataSeries}
\newcommand{\DS}{DS}

\title{DataSeries: an efficient, flexible data format for structured serial data}
% Pretend we have one author, minimizes the space we waste on that.
\numberofauthors{1} 
\author{
\alignauthor
Eric Anderson, Martin Arlitt, Brad Morrey, Alistair Veitch  \\
 \affaddr{HP Labs. 1501 Page Mill Rd.  Palo Alto, CA} \\
 \email{\{eric.anderson4, martin.arlitt, brad.morrey, alistair.veitch\}@hp.com}
}

\maketitle
% % A category with the (minimum) three required fields
% \category{H.4}{Information Systems Applications}{Miscellaneous}
% %A category including the fourth, optional field follows...
% \category{D.2.8}{Software Engineering}{Metrics}[complexity measures, performance measures]
 
% \terms{Structured serial data}

% \keywords{ACM proceedings, \LaTeX, text tagging} % NOT required for Proceedings
\section{Introduction}\label{sec:intro}

Traces, recordings and measurements taken from computer systems,
networks and scientific infrastructure are vitally important for a
large variety of tasks. In every area of computer system design,
traces from existing systems have been used to validate hypotheses,
test assumptions and estimate performance. This is true of I/O
subsystems~\cite{IORef,Ji03,Uysal03}, processor
systems~\cite{ProcRef}, network systems~\cite{NetRef} and memory
systems~\cite{MemRef}, among others. Traces and logs are also
extremely useful for fault-finding, auditing and debugging purposes
~\cite{DebugRef}. Traces composed of failure data have been used to
determine system reliability~\cite{ReliabilityRef, Schroeder07,
Pinheiro07}. Trend analyses of performance information is a core
operation of various management tools~\cite{MgmtRef}. Scientific and
medical instrumentation can also generate large amounts of
data~\cite{SciRef}, which also needs to be stored, filtered and
analyzed.

The data stored in each of these diverse uses is {\it structured
serial data}, which we define as a series of records, each record
having a specified structure (i.e., containing the same set of
variables). Structured serial data has four defining characteristics:
its structure is record-oriented; it is typically written only once,
not modified afterward, and is read many times; it is usually ordered
in some manner, e.g., chronologically; and it is typically read in a
sequential manner.  We have designed and build DataSeries, an on-disk 
data format, run-time library, and set of
tools that is optimized for analyzing this type of data.
We show that the performance of DataSeries
exceeds the performance of common trace formats and databases by at
least a factor of two, and in some cases up to an order of
magnitude. DataSeries also requires far less disk space (factors vary
from 4X to 8X in test workloads).

% which we define as an ordered series of records that share a common
% structure.  This type of data commonly occurs as trace data in
% computer systems, but since the format is essentially ordered RDBMS
% tables, the need to maintain and analyze such data occurs in a large
% number of scientific fields.

There are five key properties that are desired of a data format
and analysis system for structured serial data:

\begin{enumerate}

\item \textbf{Storage efficiency}: the data should be stored in as few
bytes as possible.

\item \textbf{Access efficiency}: accessing, interpreting and encoding
trace data, whether reading or writing, should make efficient use of
CPU and memory resources.

\item \textbf{Flexibility}: adding additional fields should not affect
users of the trace data.  Removing or modifying data fields should
only affect users who use those fields and the system should support
catching incorrect usage.  Further, the format should not constrain
the type of data being stored, and should allow multiple record types
in a single file.

\item \textbf{Self-describing}: the data set should contain the
metadata that describes the data.

\item \textbf{(Re)Usability}: the data format should have an associated
programming interface that is both expressive and easy to use.

\end{enumerate}

Although numerous tracing and measurement systems have been developed
over the last 20-30 years, we are not aware of any that meet all of
these requirements. We analyze some of these in our description of
related work (section~\ref{sec:related}).

We provide four primary contributions in this paper.  First, we
introduce DataSeries, a data format and associated library, which was
specifically designed to meet the five key properties discussed above.
Second, we discuss how DataSeries can support very large datasets
(e.g., hundreds of billions of records) on modest systems.  Third, we
describe how we have used DataSeries in practice to store a wide
variety of data types.  Fourth, we demonstrate the performance and
storage efficiency of DataSeries in a set of controlled experiments,
using empirical data sets. 

Since DataSeries software is publicly available (under a BSD software
license), and given the benefits of DataSeries that we demonstrate, we
argue that DataSeries should be considered for use by any application
that needs to store large amounts of structured serial data. Indeed
SNIA (the Storage Networking Industry Association)
has chosen DataSeries as a standard format for I/O trace data.
% and is
% currently specifying the semantics for I/O traces encoded using DataSeries.

The remainder of this paper is organized as follows.
Section~\ref{sec:related} describes the strengths and weaknesses of
existing storage technologies relative to DataSeries.
Section~\ref{sec:design} describes the design of DataSeries, including
on-disk and in-memory formats.
Section~\ref{sec:results} presents empirical and benchmark results
from our use of DataSeries to illustrate and quantify the benefits of
DataSeries. Section~\ref{sec:conclusions} concludes the
paper with a summary of our work and a list of future directions.

\section{Related Work}\label{sec:related}

We classify the related work into three categories:
those that use a customized binary format, those that use a
text-based format, and relational database systems. 

Custom binary formats are usually serialized or directly written
versions of an in-memory data structure.  As such, they usually
achieve properties 1 and 2, and fail to achieve properties 3-5,
although as we will show in the results, unless the authors are
careful they can also fail to achieve property 2.

Text formats such as column separated value often (but far from always)
achieve properties 3 and 4. XML achieves properties 3-5.  However,
they fail to achieve property 1, and can fail property two by multiple
orders of magnitude.  As we show in our results even very tuned CSV
implementations can only get to within 2-7$\times$ the
access-efficiency of DataSeries, and x-y$\times$ for the storage
efficiency.  
%{\bf TODO: need to re-do these experiments at least to
%measure the file sizes, and potentially with Tfrac text files,
%although I suspect people would use sec.usec in a block I/O trace}

Relational databases achieve properties 3, 4, and 5. RDBMS's were
designed to handle updates, so do very limited compression drastically
hurting their storage efficiency (property 1).  Our results show
$>$10x improvement on storage efficiency for DataSeries over
MySQL. 
%{\bf TODO: check the exact sizes, should be around there.}
Similarly, the generality of SQL can hurt it.  Even for fairly simple
queries running entirely on in-memeory data, DataSeries runs 2-7$\times$
faster than MySQL. 
%{\bf TODO: re-do these numbers with parallel
%decompression, etc.}  
Retrieving the data for a more complicated
calculation on the client would further slow the relative performance.

\section{Design}\label{sec:design}

DataSeries' data model is conceptually very similar to that used by
relational databases.  Logically, a DataSeries file is composed of an
ordered sequence of {\it records}, where each record is composed from
a set of {\it fields}. Each field has a {\it field-type} (e.g.,
integer, string, double, boolean) and a name. A DataSeries record is
analogous to a row in a conventional relational database. We call the
type of a row the {\it extent-type} because an {\it extent} contains a
collection of rows with the same fields and field-types. An extent is
analagous to a database table.

A single DataSeries file comprises a collection of extents
(potentially with different extent-types), plus a header and
extent-type extent at the beginning of the file, and an index extent
and trailer at the end of the file. The header on a DataSeries file 
contains the DataSeries
file version, and endianness encodings of the data types.  The
extent-type extent contains records with a single string-valued field,
each of which contains an XML specification that defines the
extent-types of all the other extents in the file. The trailer
consists of the offset and size (after compression) of the index
extent.  The offset is used to read the index extent, which has two
fields, an extent-type and an offset, to allow direct access to
extents of a single type.

The data extents themselves consist of a header, followed by the fixed
size data and the variable sized data.  Both fixed and variable sized
data may be compressed, using any one of a number of standard
compression algorithms~\cite{GZIP,BZIP,LZF,LZO}.  The header contains
metadata about the data in the extent, such as the compressed sizes of
the fixed and variable data, the number of records in the extent, the
uncompressed size of the variable length data, the compression mode,
the extent-type of the extent, and checksums of the extent before and
after compression to guard against hardware and software errors.
Checksum validation can be disabled during extent reading to improve
performance at the cost of reduced reliability.

DataSeries supports a number of options that control either the
interpretation of a field (e.g. can it be null), or the way the field
is represented prior to compression ({\em packed}, e.g. relative to
another field).  DataSeries also has some options for representation
of entire extents (e.g. ordering of the fields in a record).  
Detailed explanation of the options can be found in the DataSeries
technical report~\cite{DSTechnicalReportSnapshot}.

\section{Performance Results}\label{sec:results}

We performed various experiments to measure the effectiveness of
\DataSeries{}' compression techniques, and then further compared
\DataSeries{} to data encoding and analysis using MySQL, CSV, CStore,
Dan Ellard's NFS traces\cite{ellard03}, and the 1998 World Cup Web
traces\cite{ita08} for compression ratio and analysis execution speed.
Due to space constraints we simply give example results from our
comparison with Ellard.  More complete coverage can be found in the
DataSeries technical report~\cite{DSTechnicalReportSnapshot}.


\subsection{Ellard Traces}\label{sec:ellard}

In an effort to experiment with using DataSeries to represent and
analyze traces generated by other people, we converted the Ellard NFS
traces into DataSeries.  These traces
were originally stored as compressed text files, one record per line.
The first part of each line is a series of fixed fields, followed by a
set of key-value pairs, and finally some debugging information.  As
part of the tools, there is also a scanning program which reads the
trace files and outputs summary information for a trace.

Our evaluation came in two parts.  First, we wrote programs that
converted between the two formats.  The reversable conversion
guaranteed that we were properly preserving all of the information.
We found that the DataSeries files were on average 0.77x the size of
the original files when both were compressed using gzip.  Second, we
wrote an analysis program that implemented the first three examples in
the README that came with the tools.  We found that our analysis
program ran about 76x faster on those data files than the text
analysis program that came with the distribution.  We also found that
if we utilized lzo compression, which decompresses more quickly than
gzip, our analysis program ran about 107x faster, in exchange for
slightly larger (1.14x) data files.  This also illustrates how
DataSeries can be optimized for a given purpose (e.g. greater
compression for archival storage vs. faster decompression for more
efficient analysis). The detailed experimental setup can be found in
the DataSeries technical report~\cite{DSTechnicalReportSnapshot}.


% \subsubsection{Performance comparison}
%
% For the performance comparison, we implemented a subset of the
% analysis performed by Ellard's \texttt{nfsscan} program. In particular one
% that can perform the first three of the five example questions
% presented in the EXAMPLES file that comes with Ellard's trace tools.
% This analysis turned out to be very simple, it is just counting the
% number of requests performed of each client of each type.  We chose to
% implement this over the integer operation id, rather than the string,
% and so wrote a short table that converted NFSv2 and NFSv3 operation
% id's into a common space. The detailed experimental setup can be found in the DataSeries
% technical report~\cite{DSTechnicalReportSnapshot}.

\begin{table*}
\centering
\begin{tabular}{|c|c|c|c|c|c|c|} \hline
            & mean     & mean       & mean     & CPU     & mean     & Wall time \\
algorithm   & user (s) & system (s) & CPU (s)  & speedup & wall (s) & speedup  \\ \hline
ellard-gz   & 537.58    &  7.80     & 545.38   &  1.0x   & 545.71   &   1.0x   \\
ellard-bz2  & 638.48    & 12.68     & 651.16   &  0.836x & 571.49   &   0.955x \\
\hline
ds-gz-512k  &  22.91    &  3.62     &  26.53   & 20.557x &   7.16   &  76.186x \\
ds-gz-64k   &  21.45    &  1.14     &  22.59   & 24.147x &   5.81   &  93.945x \\
ds-gz-128k  &  23.30    &  1.19     &  24.49   & 22.268x &   6.30   &  86.604x \\
\hline
ds-bz2-16M  &  94.38    & 11.82     & 106.20   &  5.136x &  27.66   &  19.732x \\
ds-lzo-64k  &  18.71    &  1.14     &  19.85   & 27.472x &   5.10   & 106.897x \\
ds-lzo-128k &  21.15    &  1.10     &  22.25   & 24.514x &   5.74   &  95.022x \\
ds-lzo-512k &  24.07    &  4.07     &  28.14   & 19.382x &   7.40   &  73.762x \\ \hline
\end{tabular}

\caption{Summary of performance results for the two analysis programs
operating on a variety of input files.  The analysis was run over the
anon-home04-011118-* files.  For the ellard \texttt{nfsscan} program
the text files were compressed with either gz or bz2.  For the
DataSeries \texttt{ellardanalysis} program, the DataSeries files were
compressed with either gz, bz2, or lzo, and used various extent sizes
as specified.  CPU and wall time are both relative to ellard-gz.}

\label{tab:summary}
\end{table*}


Table~\ref{tab:summary} presents the summary results, showing the
impressive speedup and reduction in CPU time that can be achieved by
using DataSeries.  The different sizes specified after the compression
algorithm for the DataSeries rows are the extent sizes. 
The substantial increase in system time for dealing
with large extents for bzip2 is a result of glibc's use of mmap/munmap
for large allocations.  Every extent results in a separate pair of
mmap/munmap calls to the kernel and hence a substantial about of page
zeroing in the kernel.  
% The detailed measurements can be found 
% in the DataSeries distribution in \texttt{doc/tr/ellard-details.tex}.

\subsection{Other results}\label{sec:otherresults}

To be written

\section{Conclusions}\label{sec:conclusions}

We have described \DataSeries{}, a data format that enables the
efficient and flexible storage of structured serial data. This type of
data is used in numerous applications in all areas of computing and
science. We have identified the properties required for a system that
processes such data, and shown through a series of experiements and
comparisons to other systems that DataSeries meets satisfies these
properties. In particular, DataSeries offers significant performance
and storage efficiency benefits.

\bibliographystyle{abbrv}
{\small
\bibliography{tr-references}
}
% You must have a proper ".bib" file
%  and remember to run:
% latex bibtex latex latex
% to resolve all references
%
% ACM needs 'a single self-contained file'!
%
\balancecolumns

\end{document}
