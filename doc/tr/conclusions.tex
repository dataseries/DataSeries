\section{Conclusions}\label{sec:conclusions}

We have described \DataSeries{}, a data format that enables the efficient
and flexible storage of structured serial data. This type of data is
used in numerous applications in all areas of computing
and science, and researchers have developed almost as many ways to
store and process it as there are applications. Unfortunately, there are
many limitations to these formats. In contrast, \DataSeries{} offers five major
advantages:
\begin{enumerate}
\item \DataSeries{} improves \textit{storage efficiency}, by
 incorporating compression algorithms and related techniques.  Using
 \DataSeries{}, we have been able to store and analyze large datasets on
 a significantly smaller server and storage system than is typically
 used with databases.

\item \DataSeries{} provides much better \textit{access efficiency} than
other formats.  This enables the timely analysis of very large datasets.

\item \DataSeries{} is \textit{flexible}, in that it can handle a wide variety of
different types of data, multiple types of data in the same file, and
is easily extensible to handle new data types without changing the
format. In fact, in almost four years of use, and an increasing number of
data types, including I/O traces (disk block and NFS), batch cluster logs, 
system call traces, performance measurements and email content, we have 
not had to update the format once.

\item \DataSeries{} is \textit{self-describing}; that is, relevant metadata is retained with the data.

\item \DataSeries{} has an API to improve the \textit{usability} of the format by others.
\end{enumerate}

We have demonstrated these advantages by describing our experiences
using \DataSeries{} in a variety of situations and through a series of
experiments designed to show how \DataSeries{} compares to other types of
solutions. Based on these results,
we believe that \DataSeries{} will be useful to other researchers
who collect and analyze structured serial data.
We also think that \DataSeries{} has the potential to be
used for more than just traces and measurements of computer systems.
For example, data retention is becoming an increasingly important topic.
Legislation such as the ``Internet Stopping Adults Facilitating
the Exploitation of Today's Youth (SAFETY) Act of 2007''~\cite{SAFETY} seeks
to require ISPs to retain information on their subscribers for
extended periods of time.
% Second, in the United States the Sarbanes-Oxley (SOX) Act of 2002~\cite{SOX}
% imposed stringent new accounting and reporting practices on public
% companies.  While SOX may not directly impose data retention requirements,
% we anticipate a greater demand for the long-term archiving of structured
% serial data as a side effect.
In addition, many businesses are realizing a competitive advantage from
collecting and analyzing a wide range of data~\cite{HURD}.
The number of businesses utilizing data for such purposes
will increase over time, as others try to narrow the gap.
\DataSeries{} can potentially assist with these emerging trends.

There are a number of ways that \DataSeries{} could be enhanced in the future.
The first method would be to increase the functionality \DataSeries{} provides.
A SQL interface would allow for the easy expression (without
program development) of some types of queries and analysis.
Similarly, extending the set of generic operations that are supported
(e.g., sort-merge join) would make \DataSeries{} more appealing to some.
Several further performance enhancements are possible.
For example, different hash algorithms could be used for compressed
and uncompressed data.  Similarly, fewer sanity checks could be applied
(e.g., disable checksumming of uncompressed data except when repacking).
In other words, \DataSeries{} could provide more options to the user to
tune it to their particular performance and coherence needs.

\DataSeries{} is open source that can be downloaded from {\tt
http://tesla.hpl.hp.com/opensource/}
