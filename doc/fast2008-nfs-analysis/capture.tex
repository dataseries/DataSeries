\section{Capture}

The first stage in analyzing an NFS workload is capturing the data.
There are three places that the workload could be captured: the
client, the server, or the server.  Capturing the workload on the
clients is very parallel, but is difficult to configure and can
interfere with the actual workload.  Capturing the workload on the
server is straightfoward if the server supports capture, such as on a
unix server, or a NetApp Filer, but impacts the performance of the
server.  Capturing the workload on the network through port mirroring
is about as convinient as capture on the server, and given that most
switches implement mirroring in hardware, has no impact on network or
workload performance.  On fibre networks, even the potential impact of
port mirroring can be eliminated through the use of fiber
splitters\cite{...}. Therefore, we have always chosen to capture the
data through the use of port mirroring.

One complication with port mirroring is that usually you are mirroring
both the send and recieve directions of full-duplex ports onto a
single mirror port, meaning that the mirror port bandwidth is half the
bandwidth of the server's port.  This problem, and the related problem
of mirroring multiple ports bonded using etherchannel can be handled
on advanced switches that can selectively mirror each direction of a
particular port onto specific mirror port.  We used this functionality
in our 2003 tracing to spread 2-4 1 gigabit links over 2 1 gigabit
mirror ports.

A second complication that can occur when tracing is buffering on the
switch.  Even if the 1 second byte averages are low enough to fit onto
the mirror ports, if the switch has insufficient buffering, packets
can still be dropped.  We faced this problem in 2004 at a second
customer that used switches that only had per-port buffering rather
than shared per-card buffering as on the 2003 customer.  While
examination of the SNMP statistics led us to believe that the same
split mirroring technique we used previously would be sufficient, the
lack of buffering turned out to be a problem and drove us to use
10Gbit ethernet cards for our packet capture to reduce the need for
switch-side buffering.

A third complication for network capture is overrun of the capture
device.  While at low data rates (50-100MB/s, 30-70kpps), it is not
difficult to capture using tcpdump and standard hardware.  However, at
high data rates (800MB/s, 600-1,200kpps), traditional approaches are
insufficient.
